\documentclass[twocolumn,9pt]{article}
\usepackage{fullpage}
\usepackage{libertineRoman}
\usepackage{microtype}
\usepackage{libertineRoman}
\usepackage{eulervm}
\usepackage{sectsty}
\usepackage{notation/modes}
\usepackage{notation/lists}
\usepackage{amsmath,amssymb,amsthm}
\usepackage{proof}
\usepackage{stmaryrd}

\allsectionsfont{\sffamily}


\newtheorem{thm}{Theorem}[section]
\newtheorem{prop}[thm]{Proposition}
\newtheorem{lem}[thm]{Lemma}
\newtheorem{cor}[thm]{Corollary}

\theoremstyle{definition}
\newtheorem{definition}[thm]{Definition}
\newtheorem{example}[thm]{Example}
\newtheorem{notation}[thm]{Notation}
\newtheorem{convention}[thm]{Convention}
\newtheorem{construction}[thm]{Construction}
\theoremstyle{remark}
\newtheorem{remark}[thm]{Remark}

\numberwithin{equation}{section}

\usepackage[
  backref,
  colorlinks,
  hyperfigures,
  hyperfootnotes,
  pagebackref,
  pdfdisplaydoctitle,
  pdfencoding=auto,
  psdextra,
  unicode,
  citecolor=NavyBlue
]{hyperref}

\newcommand\JdgDecl[2]{%
  \DeclBox{\IBox{#1}}%
  \ \ \textit{pronounced}\ \ %
  \DeclBox{\OBox{\text{#2}}}%
}


\newcommand\Nat{\mathbb{N}}
\newcommand\Member[2]{\IMode{#1}\in\IMode{#2}}
\newcommand\STLC{\textbf{STLC}}
\newcommand\FStar{\ensuremath{\text{F}^\star}}
\newcommand\RedPRL{\textbf{\textcolor{red}{Red}PRL}}

\newcommand\Nil{*}
\newcommand\Fn[2]{\mathtt{fn}\ {#1}\Rightarrow{#2}}
\newcommand\DBFn[1]{\mathtt{fn} (#1)}
\newcommand\TyUnit{\mathtt{unit}}
\newcommand\TyArr[2]{#1\to{#2}}
\newcommand\Var[1]{\overline{#1}}

\newcommand\OfTy[3]{\IMode{#1}\vdash\IMode{#2}:\IMode{#3}}

\newcommand\RuleVar{\mathsf{var}}
\newcommand\RuleNil{\mathsf{nil}}
\newcommand\RuleFn{\mathsf{fn}}
\newcommand\RuleAp{\mathsf{ap}}

\newcommand\Braces[1]{{\color{gray}\left\{#1\right\}}}
\newcommand\MkGoal[3]{\langle{#1}\vdash{#2}:{#3}\rangle}
\newcommand\MkEq[2]{\langle{#1}\doteq{#2}\rangle}
\newcommand\Decomp[3]{\IMode{#1}\longrightarrow\Braces{\OMode{#2}}\sslash\Braces{\OMode{#3}}}
\newcommand\ADecomp[3]{\IMode{#1}&\longrightarrow\Braces{\OMode{#2}}\sslash\Braces{\OMode{#3}}}
\newcommand\StEval[3]{\Braces{\IMode{#1}}\sslash\Braces{\IMode{#2}}\Longrightarrow\Braces{\OMode{#3}}}
\newcommand\LocalUnify[3]{\IMode{#1}\longrightarrow\Braces{\OMode{#2}}\triangleright\Braces{\OMode{#3}}}
\newcommand\ALocalUnify[3]{\IMode{#1}&\longrightarrow\Braces{\OMode{#2}}\triangleright\Braces{\OMode{#3}}}
\newcommand\NoOccurs[2]{\IMode{#1}\mathrel{\text{does not occur in}}\IMode{#2}}
\newcommand\Unify[3]{\Braces{\IMode{#1}}\triangleright\Braces{\IMode{#2}}\Longrightarrow\Braces{\OMode{#3}}}
\newcommand\InferTy[3]{\IMode{#1}\vdash\IMode{#2}\Rightarrow\OMode{#3}}
\newcommand\Subst[2]{\widehat{#1} (#2)}




\title{Learning Simple Types}
\author{Costin B\u{a}descu\and Ankush Das\and Ryan Kavanagh\and Jon Sterling}
\date{}

\begin{document}
\maketitle

\section{Introduction}

Type structure is the fundamental organizing principle of code, both
in theory and in practice. Types express the range of significance of
a piece of code: for instance, the function $\IMode{\Fn{x}{x+1}}$ is
meaningful only when applied to an expression which supports an
addition operation.

Depending on the purpose of a programming language, different forms of
type structure may be imposed; suitable implementation strategies for
type systems vary wildly depending on their characteristics.

\begin{enumerate}
\item Some type systems (including \emph{simple types} and
  \emph{prenex-polymorphic types}) support \textbf{type inference}: an
  algorithm which either assigns a type to a piece of code, or
  exhibits that code as ill-typed.
\item More sophisticated type systems (including full
  \emph{polymorphic types} and \emph{intensional dependent types})
  support \textbf{type checking}: an algorithm which decides whether a
  piece of code exhibits a \emph{given} type.
\item Even more sophisticated type systems (such as \emph{extensional
    dependent types}) support only \textbf{type proving}: given a
  piece of code, a type, and a candidate derivation (proof) that this
  code exhibits this type, we can decide whether this derivation is
  correct.
\end{enumerate}

As can be seen above, available implementation strategies dwindle as
type systems become more sophisticated, eventually converging at a
state of affairs where a practitioner must supply a proof of
well-typedness. However, this does not rule out the use of
\emph{heuristics} in sophisticated type systems which achieve type
inference or type checking in some cases, but fail in others.

In some implementations of dependent type theory such as
Nuprl~\cite{constable:1986} and \RedPRL{}~\cite{redprl:2016}, these
heuristics take the form of \emph{proof
  tactics}~\cite{gordon-milner-wadsworth:1979,paulson:1987,gordon:2000}. It
is also possible to consider heuristics based on external SMT solvers
as in the \FStar{}~language~\cite{ahman-et-al:2017}, which uses the Z3
solver to resolve obligations. Finally, machine learning techniques
have also been applied to interactive proof assistants such as
Coq~\cite{komendantskaya-heras-grov:2012}.

\paragraph{Thesis}
We believe that at the current stage of scientific development, where
fully-automated inference is no longer possible, the principal
contradiction of type systems praxis lies in the development of
effective heuristics for type inference and type checking. To this
end, we have applied techniques from \emph{deep learning} to address a
restricted subproblem, namely type inference for the simply typed
lambda calculus (\STLC{}) with a single base type.

\section{Simply Typed Lambda Calculus}
The simply typed lambda calculus (\STLC{}) is the simplest possible
programming language, and provided a suitable environment in which to
test our ideas before applying them to more difficult problems. In
Figure~\ref{fig:stlc-grammar-and-declarative-typing}, we define the
syntax and \emph{declarative} typing rules for the \STLC{}.

\paragraph{Declarative vs algorithmic typing}
Declarative typing judgments give a mathematical definition of the
typing relation for programs; such a definition is not usually
suitable for implementation. In the case of \STLC{}, we will define an
algorithmic \emph{type inference} relation which is adequate (sound
and complete) with respect to the declarative typing relation.

\begin{figure*}
  \begin{align*}
    \IMode{\tau} &::= \FormatList{\OMode}{\mid}{\TyUnit,\TyArr{\IMode{\tau}}{\IMode{\tau}}}
    \tag{Types}
    \\
    \IMode{e} &::= \FormatList{\OMode}{\mid}{x, \Nil, \Fn{\IMode{x}}{\IMode{e_x}}, \IMode{e}(\IMode{e})}
    \tag{Programs}
    \\
    \IMode{\Gamma} &::= \FormatList{\OMode}{\mid}{\cdot,{\IMode{\Gamma},\IMode{x}:\IMode{\tau}}}
    \tag{Typing contexts}
  \end{align*}
  \begin{gather*}
    \JdgDecl{
      \OfTy{\Gamma}{e}{\tau}
    }{
      $\IMode{e}$ is of type $\IMode{\tau}$ in context $\IMode{\Gamma}$
    }
    \\[6pt]
    \infer[\RuleVar]{
      \OfTy{\Gamma,x:\tau,\Gamma'}{x}{\tau}
    }{
    }
    \qquad
    \infer[\RuleNil]{
      \OfTy{\Gamma}{\Nil}{\TyUnit}
    }{
    }
    \\[6pt]
    \infer[\RuleFn]{
      \OfTy{\Gamma}{\Fn{x}{e_x}}{\TyArr{\tau_1}{\tau_2}}
    }{
      \OfTy{\Gamma,x:\tau_1}{e_x}{\tau_2}
    }
    \qquad
    \infer[\RuleAp]{
      \OfTy{\Gamma}{e_1(e_2)}{\tau_2}
    }{
      \OfTy{\Gamma}{e_1}{\TyArr{\tau_1}{\tau_2}}
      &
      \OfTy{\Gamma}{e_2}{\tau_1}
    }
  \end{gather*}
  
  \caption{Grammar and declarative typing rules for the \STLC{}. In
    this definition, we use names for bound variables and indicate
    binding scope using
    subscripts.}\label{fig:stlc-grammar-and-declarative-typing}
\end{figure*}

\subsection{Algorithmic type inference}

In Figure~\ref{fig:stlc-algorithmic-grammar}, we give a version of the
grammar of \STLC{} augmented with \emph{type metavariables}, which
play a crucial role in the type inference process.
Our grammar also includes syntax
for the states of the type inference machine, including typing goals,
typing goal stacks, unification equations, unification constraint
sets, and type substitutions.

Additionally, in this treatment we use De Bruijn indices rather than
names to account for binding; De Bruijn indices reflect the
address/location of a variable's binding in a syntax tree.
%
Among other things, De Bruijn presentations of calculi with binding
structure are easier to work with computationally, because they
eliminate the need to consider equivalence classes of terms under
permutations of bound variable names.

\begin{figure*}
  \begin{align*}
    \IMode{\tau} &::= \FormatList{\OMode}{\mid}{\alpha, \TyUnit,\TyArr{\IMode{\tau}}{\IMode{\tau}}}
    \tag{Types}
    \\
    \IMode{e} &::= \FormatList{\OMode}{\mid}{\Var{n}, \Nil, \DBFn{\IMode{e}}, \IMode{e}(\IMode{e})}
    \tag{Programs}
    \\
    \IMode{\Gamma} &::= \FormatList{\OMode}{\mid}{\cdot,{\IMode{\Gamma},\IMode{\tau}}}
    \tag{Typing contexts}
    \\
    \IMode{\mathcal{G}} &::= \OMode{\MkGoal{\IMode{\Gamma}}{\IMode{e}}{\IMode{\tau}}}
    \tag{Typing goals}
    \\
    \IMode{\mathcal{S}} &::= \FormatList{\OMode}{\mid}{\cdot, {\IMode{\mathcal{G}},\IMode{\mathcal{S}}}}
    \tag{Typing stacks}
    \\
    \IMode{\mathcal{E}} &::= \OMode{\MkEq{\IMode{e}}{\IMode{e}}}
    \tag{Unification equations}
    \\
    \IMode{\mathcal{C}} &::= \FormatList{\OMode}{\mid}{\cdot,{\IMode{\mathcal{E}},\IMode{\mathcal{C}}}}
    \tag{Unification constraints}
    \\
    \IMode{\gamma} &::= \FormatList{\OMode}{\mid}{\cdot, {\IMode{\alpha}\hookrightarrow\IMode{\tau},\IMode{\gamma}}}
    \tag{Substitutions}
  \end{align*}

  Write $\IMode{\Subst{\gamma}{-}}$ for the action of the substitution
  $\IMode{\gamma}$, which proceeds by replacing $\IMode{\alpha}$ with
  $\OMode{\tau}$ when $\IMode{\alpha\hookrightarrow\tau}$ is in
  $\IMode{\gamma}$. This notation applies at all syntactic categories.

  \caption{Grammar of algorithmic \STLC{} used in algorithmic type
    inference.  For this presentation, we use \emph{de Bruijn indices}
    $\IMode{\Var{n}}$ with $\Member{n}{\Nat}$ rather than names for
    program variables. Observe that we have added \emph{metavariables}
    $\IMode{\alpha}$ to the grammar of types, which will serve as
    placeholders to be resolved during unification.
  }\label{fig:stlc-algorithmic-grammar}
\end{figure*}

Next, in Figure~\ref{fig:algorithmic-type-inference} we give the full
type inference algorithm for \STLC{} in terms of constraint generation
and first-order unification. We have used this algorithm in our
implementation in order to generate training data for our neural
network.

\begin{figure*}
  \begin{gather*}
    \JdgDecl{
      \Decomp{\mathcal{G}}{\mathcal{S}}{\mathcal{C}}
    }{
      $\IMode{\mathcal{G}}$ decomposes into $\OMode{\mathcal{S}}$ modulo $\OMode{\mathcal{C}}$
    }
    \\[6pt]
    \begin{aligned}
      \ADecomp{
        \MkGoal{\Gamma}{\Var{n}}{\tau}
      }{
        \cdot
      }{
        \MkEq{\tau}{\Gamma_n}
      }
      \\
      \ADecomp{
        \MkGoal{\Gamma}{\Nil}{\tau}
      }{
        \cdot
      }{
        \MkEq{\tau}{\TyUnit}
      }
      \\
      \ADecomp{
        \MkGoal{\Gamma}{\DBFn{e}}{\tau}
      }{
        \MkGoal{\Gamma,\alpha}{e}{\beta}
      }{
        \MkEq{\tau}{\TyArr{\alpha}{\beta}}
      }
      \\
      \ADecomp{
        \MkGoal{\Gamma}{e_1(e_2)}{\tau}
      }{
        \MkGoal{\Gamma}{e_1}{\TyArr{\alpha}{\tau}},
        \MkGoal{\Gamma}{e_2}{\alpha}
      }{
        \cdot
      }
    \end{aligned}
  \end{gather*}

  \begin{gather*}
    \JdgDecl{
      \StEval{\mathcal{S}}{\mathcal{C}}{\mathcal{C}'}
    }{
      $\IMode{\mathcal{S}}$ modulo $\IMode{\mathcal{C}}$ generates constraints $\OMode{\mathcal{C}'}$ 
    }
    \\[6pt]
    \infer{
      \StEval{\cdot}{\mathcal{C}}{\mathcal{C}}
    }{
    }
    \qquad
    \infer{
      \StEval{\mathcal{G},\mathcal{S}}{\mathcal{C}}{\mathcal{C}'}
    }{
      \Decomp{\mathcal{G}}{\mathcal{S}_{\mathcal{G}}}{\mathcal{C}_{\mathcal{G}}}
      &
      \StEval{\mathcal{S}\cup\mathcal{S}_{\mathcal{G}}}{\mathcal{C}\cup\mathcal{C}_{\mathcal{G}}}{\mathcal{C}'}
    }
  \end{gather*}

  \begin{gather*}
    \JdgDecl{
      \LocalUnify{\mathcal{E}}{\mathcal{C}}{\gamma}
    }{
      $\IMode{\mathcal{E}}$ generates additional constraints $\OMode{\mathcal{C}}$ and substitution $\OMode{\gamma}$
    }
    \\[6pt]
    \infer{
      \LocalUnify{
        \MkEq{\alpha}{\tau}
      }{
        \cdot
      }{
        \alpha\hookrightarrow\tau
      }
    }{
      \NoOccurs{\alpha}{\tau}
    }
    \qquad
    \infer{
      \LocalUnify{
        \MkEq{\tau}{\alpha}
      }{
        \cdot
      }{
        \alpha\hookrightarrow\tau
      }
    }{
      \NoOccurs{\alpha}{\tau}
    }
    \\[6pt]
    \begin{aligned}
      \ALocalUnify{\MkEq{\TyUnit}{\TyUnit}}{\cdot}{\cdot}
      \\
      \ALocalUnify{\MkEq{\TyArr{\tau_1}{\tau_2}}{\TyArr{\tau_1'}{\tau_2'}}}{
        \MkEq{\tau_1}{\tau_1'},
        \MkEq{\tau_2}{\tau_2'}
      }{\cdot}
    \end{aligned}
  \end{gather*}

  \begin{gather*}
    \JdgDecl{
      \Unify{\gamma}{\mathcal{C}}{\gamma'}
    }{
      constraints $\IMode{\mathcal{C}}$ take substitution $\IMode{\gamma}$ to $\OMode{\gamma'}$
    }
    \\[6pt]
    \infer{
      \Unify{\gamma}{\cdot}{\gamma}
    }{
    }
    \qquad
    \infer{
      \Unify{\gamma}{\mathcal{E},\mathcal{C}}{\gamma'}
    }{
      \LocalUnify{\mathcal{E}}{\mathcal{C}_{\mathcal{E}}}{\gamma_{\mathcal{E}}}
      &
      \Unify{\gamma\cup\gamma_{\mathcal{E}}}{\Subst{\gamma_{\mathcal{E}}}{\mathcal{C}}\cup\mathcal{C}_{\mathcal{E}}}{\gamma'}
    }
  \end{gather*}

  \begin{gather*}
    \JdgDecl{
      \InferTy{\Gamma}{e}{\tau}
    }{
      the term $\IMode{e}$ infers type $\OMode{\tau}$ given context $\IMode{\Gamma}$
    }
    \\[6pt]
    \infer{
      \InferTy{\Gamma}{e}{\Subst{\gamma}{\alpha}}
    }{
      \StEval{
        \MkGoal{\Gamma}{e}{\alpha}
      }{\cdot}{\mathcal{C}}
      &
      \Unify{\cdot}{\mathcal{C}}{\gamma}
    }
  \end{gather*}
  
  \caption{Definition of algorithmic type inference for \STLC{}. We
    have factored the type inference algorithm into several judgments:
    local decomposition of typing problems, evaluation of typing
    problem stacks into constraint sets, local decomposition of
    equational constraints, and evaluation of constraint sets into
    substitutions.}\label{fig:algorithmic-type-inference}
\end{figure*}


\section{Training Data Generation}

Ryan, please write something here.

\section{Neural Network and Results}

Ankush and/or Costin, please write something here.

\bibliographystyle{abbrv}
\bibliography{references/refs}

\end{document}