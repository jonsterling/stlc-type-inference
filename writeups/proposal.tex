\documentclass{article}
\usepackage[left=3cm,top=2cm,right=3cm,nohead,bottom=3cm]{geometry}
\usepackage{listings}

\title{Type Inference using Deep Learning\\15-780 Project Proposal}
\author{Ankush Das, Costin B\u{a}descu, Jonathan Sterling and Ryan Kavanagh}

\begin{document}
\maketitle

\section{Problem and Motivation}
Type inference refers to the deduction of the type of an expression in a
typed programming language. Functional programming
languages, such as Standard ML, OCaml or Haskell, can automatically infer the
type of an expression even in the absence of type annotations.  This
form of type inference is usually implemented using the well-known
Hindley--Milner algorithm~\cite{hindley1969principal, milner1978theory}.
Unfortunately, in more complex type systems (e.g. System
F~\cite{wells1999typability}), type inference is often an undecidable
problem.

To illustrate how type inference works, consider the following OCaml program:
\begin{lstlisting}[columns=fullflexible, language=Caml]
let rec factorial n =
	if (n = 0) then 1
	else n * factorial (n-1);;
\end{lstlisting}
OCaml's type checker will infer from the comparison $\mathsf{(n = 0)}$
that $\mathsf{n}$ must have the same type as $0$, viz. $int$. Since the
result of $\mathsf{factorial \; (n-1)}$ is multiplied by $\mathsf{n}$,
it follows that \emph{factorial} must also return a term of type
$int$. Hence, the type checker infers that the \emph{factorial} function
has the type $int \to int$.

\section{Outline of Approach}
We propose to use deep learning to infer the types of expressions in the
simply-typed $\lambda$-calculus (STLC). Since type inference for this
calculus is decidable, we can use existing algorithms, such as
Hindley--Milner, to generate a large training set of
$\lambda$-expressions paired with their corresponding
types. Furthermore, because type checking for the STLC is decidable, we
can determine whether or not the type output by the neural network for a
given expression is correct or not. The $\lambda$-calculus provides a
tractable first step in exploring the applicability of deep learning
techniques to programming language frameworks. Conditioned upon the
succcess of this project, future extensions include the generalization
to more complex type systems (e.g. System F, dependent type
theory~\cite{martin-lof:1984}) and even more sophisticated undecidable
problems such as program synthesis.

\bibliographystyle{abbrv}
\bibliography{refs}

\end{document}